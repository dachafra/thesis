
\chapter{Objectives and Contributions}
\label{chap:objectives}
The main goal of this work is the use of declarative mapping rules and their knowledge encoded for the construction and evaluation of knowledge graphs from heterogeneous data sources. This chapter presents the objectives and contributions to the state of the art of our work. Additionally, we detail the assumptions considered when we started this work, hypothesis and restrictions that delimit and describe the scope of this thesis.

\section{Objectives}
In the context of this thesis we identify two different goals. First, we provide a definition and a set of properties over the new concept of \textit{mapping translation} and we describe two approaches that take into account this concept to improve the construction and access of knowledge graphs from heterogeneous data sources. The second goal of the thesis consists in defining the main steps and resources for an objective and proper evaluation of these kind of approaches.

In order to achieve the first objective of this thesis, the open research problems that have to be solved are:

\begin{itemize}
    \item The construction of knowledge graphs (virtual or materialized) from heterogeneous data sources using declarative mapping rules is still an open issue in the state of the art. Based on the W3C recommendation R2RML~\citep{R2RML}, and with the aim of providing support to other format beyond relational databases, many new declarative mapping languages are proposed such as RML~\citep{dimou2014rml}, xR2RML~\citep{michel2015translation}, KR2RML~\citep{slepicka2015kr2rml}, CSVW~\citep{tennison2015model}, or D2RML~\citep{chortaras2018mapping}. 
    % PC: Rephrase: instead of _many_ I would say that “One of the motivations to build such declarative mapping languae, is to ...
    Many of these extensions are proposed with the aim of solving specific heterogeneity issues of the input data, hence, they start to lose their generalization and the benefit of the declarative approach of the rules. To deal with these problems, we propose the possibility of the translation among these different languages, covering at least those common characteristics that are shared across languages, introducing one of the main ideas about a new generation of knowledge graph construction approaches.
    \item The construction of virtualized knowledge graphs from raw data (e.g., CSV, JSON or XML) has been tackled as an engineering process delegating the management of these data sources to databases such as Apache Drill\footnote{\url{https://drill.apache.org/}} and Spark-SQL\footnote{\url{https://spark.apache.org/sql/}}. These systems allow the generation of a simple relational database layer over the input data sources. However, in order to tackle the advantage of proposed SPARQL-to-SQL optimization techniques~\citep{priyatna2014formalisation,calvanese2017ontop}, a well-formed RDB is required, including its corresponding constraints. Focused on the specific case of tabular data, we propose a framework that exploits and translates mapping rules and metadata annotations to deal with the typical heterogeneity issues for querying these datasets under an OBDA environment. Its main aim is to improve query evaluation performance, as well as query completeness.
    \item Multiple types of query interfaces have been proposed to query heterogeneous data on the web. One of the most recent and relevant incorporation is GraphQL~\citep{graphql}. This technology aims to solve problems such as under/over fetching~\citep{bryant2017graphql,vogel2017experiences,mukhiya2019graphql} of other common used interfaces 
    %PC: “REST APIs” here is way too broad. The idea of LDF also solves this problem, yet it does follow the REST architectural style. https://speakerdeck.com/pietercolpaert/graphql-vs-rest -- I would thus rephrase REST APIs to something like: subject page-style HTTP APIs?
    (e.g., REST APIs). This specification has been included by multiple organizations in their systems as an integrating query layer over multiple data sources, hence, creating a bridge between GraphQL and knowledge graph construction technologies is relevant. Although most of the proposed approaches are focused on understand how to translate GraphQL to SPARQL queries, we propose a framework that translates declarative mapping rules to functional programmed GraphQL data wrappers (known as GraphQL resolvers) help in their creation and ensuring that the target data model uses common and shared vocabularies.
\end{itemize}

The previous methodological goals have two associated technological goals. The first one is the implementation of a virtual knowledge graph creation system over tabular data (Morph-CSV). The second one is implemented by Morph-GraphQL, that translates R2RML mapping rules to programmed GraphQL resolvers for accessing legacy relational database instances.
The second goal is focused on proposing evaluations of knowledge graph construction systems to understand what are their main current limits, and it has the following open research problems:
\begin{itemize}
    \item There is currently no way to obtain objective information about the conformance of knowledge graph construction engines with specifications of declarative mapping languages that go beyond relational databases. Based on the main extension of R2RML, RML~\citep{dimou2014rml}, we proposed a set of representative test cases covering other types of data formats such as JSON, CSV and XML.
    \item There is no analysis of what the parameters are that affect the process of constructing a knowledge graph from heterogeneous data sources. Previous evaluations have only took into account data size as a relevant parameter~\citep{lefranccois2017sparql,csimcsek2019rocketrml}. We identify the set of parameters, most of them extracted from the mapping rules, that can also affect to the behaviour of these engines.
    \item Finally, the current knowledge graph construction benchmarks~\citep{lanti2015npd,bizer2009berlin} that mainly focused on relational databases, are not enough to test the capabilities of the new generation of engines that are able to query (??) materialized or virtualized data in several formats. A representative benchmark to test the capabilities of these approaches with objective and useful information is defined.
\end{itemize}



\section{Contributions to the State of the Art}


\begin{enumerate}
    \item[\textbf{C1.1.}]  Mapping Translation: concept and properties
    \item[\textbf{C1.2.}] Morph-CSV 
    \item[\textbf{C1.3.}] Morph-GraphQL
    \item[\textbf{C2.1.}] Definition of a set of test cases to test the conformance of KG construction engines in RML.
    \item[\textbf{C2.2.}] Identification and experimental evaluation of what are the parameters
    \item[\textbf{C2.3.}] Complete benchmark for evaluating
\end{enumerate}

\section{Assumptions}

\begin{enumerate}[label=\textbf{A{\arabic*}}]
    \item Mapping rules and annotations are declarative and following W3C standards (or extend them). 
    \item The target schema for integrating the source data is already created.
    \item Mapping rules and metadata annotations are available.
    \item Data are represented in different formats but not in RDF.
\end{enumerate}

\section{Hypothesis}

\begin{enumerate}[label=\textbf{H{\arabic*}}]
    \item It is possible to translate declarative mappings
    \item Virtualized Knowledge Graphs are improved in terms of completeness and performance over raw data exploiting declarative mapping rules and metadata annotations.
    \item Not only size data is relevant in the evaluation of KG construction engines
    \item A benchmark blablabla is able to stress and provide a full overview of the state of different engines 
\end{enumerate}

\section{Restrictions}

\begin{enumerate}[label=\textbf{R{\arabic*}}]
    \item Data are located in the same physical place.
    \item We restrict declarative mapping rules following RML+FnO~\citep{de2017declarative} specification and metadata annotations in CSVW for Morph-CSV and R2RML for Morph-GraphQL.
    \item The source data for evaluating KG construction systems do not need to be transformed.
\end{enumerate}
