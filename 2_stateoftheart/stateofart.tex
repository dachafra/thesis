\chapter{State of the Art}
\label{chap:soa}

In this chapter, we introduce the current state of the art in knowledge graph construction using declarative mapping rules. We provide an overview of approaches, techniques and methodologies for constructing and querying (virtual) knowledge graphs based on semantic web technologies. We describe the declarative annotations and mapping languages specifications that have been proposed to construct these kind of data models together with their main benefits. Finally, we present the current methodologies to evaluate the quality of knowledge graph construction engines such as benchmarkings and test-cases.


\section{Ontology-Based Data Access and Integration}


\section{Representation and Query Language for the Semantic Web}

\subsection{RDF: Resource Description Framework}

\subsection{SPARQL}




\section{Annotations in Knowledge Graph Construction}
One of the main components for the construction of knowledge graphs are the annotations. Additionally to the mapping rules, that relate the target model with the input sources in a typical data integration system definition, we include in the annotations set the constraints concept. In a DIS, the constrains property allows to: i) define ad-hoc transformation functions that permit the cleaning and preparation of the input data; ii) definition of metadata annotations to describe the content of the input source. This property is essential during a knowledge graph construction process as it is able to deal with the typical features of heterogeneous data sources such as the absence of a well-defined and fixed data schema, a normalized database instance or the non-explicit declarations of relations among the sources. We start this section discussing existing approaches for the design of mappings. Then, we describe the current mapping language specifications and their standardization through the W3C. Finally, we present approaches to define, declaratively, constraints over a DIS.

\subsection{Mapping Rules}
The mapping layer contains information about how the input sources are related with the target model. There are two basic approaches for defining mapping rules in a data integration system: Local as a View (LAV) and Global as a View (GAV). In semantic web, the usual followed approach to define this rules is the Global as View one. We now provide a description of each proposals more in detail.

\subsubsection{Local as a View Mapping rules (LAV)}
In \citep{ullman1997information} the elements of the source schema $S$ are mapped  to a query $Q_G$ over the target schema $G$. The main benefits of this approach is that it supports continuous changes of the source schema (e.g., adding new sources or modify their underlying representation) since there is no need to change the query processing component. Thus, LAV is usually useful when the global schema $G$ is stable but the local schema $S$ may suffer modifications over the time. However, one of its main disadvantages is that cannot represent source $S$ information if it is not modeled in the global schema, hence, the approach usually provides partial answers for a query $Q_G$. Query translation following this approach is not a trivial process, as the $Q_G$ has to be translated to an equivalent query over the source schema $S$. These techniques are usually known as query translation using views~\citep{halevy2001answering}.

ToDo: ADD EXAMPLE

\subsubsection{Global as a View Mapping Rules (GAV)}
In \citep{halevy2001answering} each element of the global schema $G$ is mapped to a query over $Q_S$ the source schema $S$. Opposed to LAV approach, the benefits of following a GAV approach is that it supports changes over the global schema $G$, as the queries are defined following the source schema $S$. Although there are no theoretical limitations to provide access to other data formats, ontology-based data integration processes have been traditionally focused on allowing the integration of relational databases as source schema, based on SPARQL-to-SQL translation techniques. Due to the aforementioned limitations in these techniques for LAV approaches, most of the semantic web mapping rules specifications following the GAV approach (e.g., R$_2$O, DR2Q, R2RML). 

ToDO: ADD EXAMPLE

\subsubsection{R2RML: W3C Recommendation}

\subsubsection{RML: Extending R2RML for Heterogeneous Data}

\subsection{Declarative Constraints: Transformation Functions and Metadata}

\subsubsection{Transformation Functions in Mappings}

\subsubsection{Metadata for data on the web}


\section{Knowledge Graph Construction Engines}

\section{Evaluation of Knowledge Graph Construction}

\subsection{Test-Cases}
\subsection{Benchmarks}



\section{Conclusions}
