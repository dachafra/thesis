\chapter{State of the Art}
\label{chap:soa}

In this chapter, we introduce the current state of the art in knowledge graph construction using declarative mapping rules. We provide an overview of approaches, techniques and methodologies for constructing and querying (virtual) knowledge graphs based on semantic web technologies. We describe the declarative annotations and mapping languages specifications that have been proposed to construct these kind of data models together with their main features. We also present the current methodologies to evaluate the quality of knowledge graph construction engines such as benchmarkings and test-cases. 

More in detail, in Section \ref{sec:soa_integration} we describe the foundations of data integration concept and its relation with the theoretical contributions where the global schema is defined by an ontology. Section \ref{sec:soa_representation} summarizes the semantic web technologies used for representing and querying data (i.e., the RDF model and the SPARQL query language). Section \ref{sec:soa_annotations} presents an overview of the most important contributions for representing declarative annotations using semantic web technologies, where we are focused on the representations of mapping rules and data constraints. Additionally, Section \ref{sec:soa_engines} presents knowledge graph construction systems and their corresponding optimizations while Section \ref{sec:soa_evaluations} finally, describes the evaluation methodologies for these approaches. We conclude in Section \ref{sec:soa_conclusions} with a set of motivations and gaps that will help to understand the contributions of this thesis.


\section{Data Integration and OBDA}
\label{sec:soa_integration}
A data integration system (DIS) is defined as the process of making available a set of different sources through a common view~\citep{Lenzerini02}. Formally, it is described as $DIS$ = $\langle \mathcal{G}, \mathcal{S}, \mathcal{M} \rangle$, where:
\begin{itemize}
    \item $\mathcal{G}$ is the \textit{global schema}, expressed in a language $\mathcal{L_G}$ over an alphabet $\mathcal{A_G}$.
    
    \item $\mathcal{S}$ is the \textit{source schema}, expressed in a language $\mathcal{L_S}$ over an alphabet $\mathcal{A_S}$.
    
    \item $\mathcal{M}$ is the \textit{mapping} between $\mathcal{G}$ and $\mathcal{S}$, constituted by a set of assertions matching queries over $\mathcal{S}$ and $\mathcal{G}$, in order to establish correspondences between concepts in both schemes.
\end{itemize}

Different types of data integration systems have been proposed. Data warehouses~\citep{vassiliadis2009survey} are proposed to integrate multiple and heterogeneous data sources in a centralized place, which is usually known in enterprises ecosystems as an \textit{extract-transform-load} process (ETL). In contrast, mediators~\citep{wiederhold1992mediators} have been proposed as another data integration approach where the data remains in the data sources. For accessing the data, queries defined over the global schema are translated to the source schema and executed. There are multiple systems that implement these ideas with their corresponding optimizations~\citep{tsimmis1994,rajaraman1996querying,roth1997don}.

Ontology based data access (OBDA) and integration (OBDI) are data integration systems where the global schema is defined by an ontology~\citep{poggi2008linking}. The formal framework presented in~\citep{xiao2018obdasurvey} defines an OBDA specification as a tuple $P$ = $\langle O,S,M\rangle$ where $O$ is an ontology, $S$ is the source schema, and $M$ a set of mappings. Additionally, an OBDA instance is defined as a tuple $PI$ = $\langle P,D\rangle$ where P is an OBDA specification and $D$ is a data instance conforming to $S$. The main difference between OBDA and OBDI systems is that in OBDA $D$ is fixed to an specific data format, while OBDI extends $D$ to cover heterogeneous data sources and formats. In both ontology-based approaches, two different alternatives exist to enable data access: (1) those where data are materialized taking into account the mappings and the ontologies which can be seen as a ontology-based ETL process, and (2) those where the transformation is done on the queries, which can then be evaluated on the original data sources, so, it can be defined as a new kind of mediator system. In this work we refer to the first alternative as   a materialized knowledge graph construction process, while the second alternative is defined as a virtual knowledge graph construction one. As we are focused on creating KG from heterogeneous data sources, both alternatives can be described as an OBDI approach.

\section{Representation and Query Language for the Semantic Web}
\label{sec:soa_representation}
In this section, we provide a overview of the core semantic web technologies that have a relevant role during a knowledge graph construction process. We discuss in detail the Resource Description Framework (RDF)~\citep{RDF}, used for modeled the target data model in the DIS (i.e., the ontology), but also usually used for declaring the mapping rules~\citep{R2RML,dimou2014rml,michel2015translation}. We also provide a detailed explanation of the SPARQL query language~\citep{SPARQL} as it is an essential input for the virtual knowledge graph construction techniques (e.g., translating SPARQL to SQL).

\subsection{RDF: Resource Description Framework}
The Resource Description Framework (RDF)~\citep{RDF} is the basic data model used in the semantic web. Basically, RDF is defined by a set of \textit{triples}, where each triple is represented in the form of $<s,p,o>$ where $s$ is the subject, $p$ the predicate and $o$ the object. One of the main features of RDF is the used of URIs to represent concepts and the relationships between them. This idea allows to have unique identifiers across the web for each resource. Assuming that there are a pairwise disjoint infinite $I$, $B$ and $L$ (IRIs, blank nodes and literals respectively), we can formally defined an RDF triple as $(s,p,o) \in (I \cup B) \times I \times (I \cup B \cup L)$. Additionally, we can defined $G$ as an RDF graph composed by a set of RDF triples.

Listing \ref{list:soa_rdf_example} shows an example of an RDF graph with a set of triples in the transport domain (shown in its graphical representation in Figure \ref{fig:soa_rdf_example}). As we can observe, there are, in total, 4 different triples (or statements), all of them with the same subject, defined using the URI \textit{http://transport.org/trip/1}. We can also observe that all the predicates of the triples (e.g., http://vocab.gtfs.org/terms\#stop) are also defined by an URI. More in detail, the first triple declares that \textit{http://transport.org/trip/1} is a trip defined in the LinkedGTFS vocabulary\footnote{\url{http://vocab.gtfs.org/terms}}. The predicate \texttt{rdf:type} is used in the RDF data model for classify the resources. Then, predicates \texttt{gtfs:shortName} and \texttt{gtfs:wheelchair} represent the information about the name of the trip and if it is accessible or not using a wheelchair. The objects of these two triples are literals in RDF that can be either simple string (``Puerta del Sur'') or typed literal (``false''$\hat{\;}\hat{\;}$xsd:boolean). The typed literals contain data values with a tag for representing its data type. Finally, \texttt{gtfs:stop} indicates one of the trips stops, using and URI, which means that the stop is also a resource. 


\begin{lstlisting}[float,caption=Example of RDF graph,frame=tlrb,label={list:soa_rdf_example}, columns=fullflexible]
@prefix gtfs: <http://vocab.gtfs.org/terms#> .
@prefix xsd: <http://www.w3.org/2001/XMLSchema#> .

<http://transport.org/trip/1> rdf:type gtfs:Trip .
<http://transport.org/trip/1> gtfs:stop <http://transport.org/stop/360> .
<http://transport.org/trip/1> gtfs:shortName "Puerta del Sur" .
<http://transport.org/trip/1> gtfs:wheelchair "false"^^xsd:boolean .
\end{lstlisting}

\begin{figure}[!h]
\centering
\includegraphics[width=\textwidth]{figures/state-of-the-art/RDF.pdf}
\caption{Graphical representation of an RDF Graph}
\label{fig:soa_rdf_example}
\end{figure}


The W3C provide several formats to serialize an RDF graph. RDF/XML\footnote{\url{https://www.w3.org/TR/rdf-syntax-grammar/}} is the first serialization proposed and it is supported by XML. Notation 3 (N3) provides a human readable serialization. N-Triples\footnote{\url{https://www.w3.org/TR/n-triples/}} and Turtle\footnote{\url{https://www.w3.org/TR/turtle/}} are subsets of N3, widely adopted in the semantic web. Finally, the last serialization that has been incorporated as W3C recommendation is JSON-LD\footnote{\url{https://www.w3.org/TR/json-ld11/}}. Since it is based on JSON, it facilitates the consumption of RDF data to developers and practitioners as it encapsulates the RDF in a standard JSON document. There are other serialization of RDF that are not recommended by W3C but are highly adopted for specific purposes. For example, HDT~\citep{fernandez2013binary} is a binary serialization of RDF for publishing and exchanging RDF data at large scale.

\begin{lstlisting}[float,caption=Example of SPARQL query,frame=tlrb,label={list:soa_sparql_example}, columns=fullflexible]
PREFIX gtfs: <http://vocab.gtfs.org/terms#> .

SELECT ?name WHERE {
    ?trip rdf:type gtfs:Trip .
    ?trip gtfs:shortName ?name .
}
\end{lstlisting}
\subsection{SPARQL}
SPARQL is the W3C recommendation graph matching query language for RDF graphs. For defining the syntax of the query language we use the work presented in~\citep{perez2009semantics}, which defines the SPARQL graph patterns recursively as follows:
\begin{itemize}
    \item A tuple $(I \cup V) \times (I \cup V) \times (I \cup L \cup V)$ is a graph pattern (triple pattern if it is single), where $I$ is an IRI, $V$ is a variable and $L$ a literal.
    \item If $P_1$ and $P_2$ are graph patterns, then expression ($P_1$ AND $P_2$), ($P_1$ OPT $P_2$) and ($P_1$ UNION $P_2$) are also graph patterns.
    \item If $P$ is a graph pattern and $R$ is a SPARQL built-in condition, then ($P$ FILTER $R$) is also a graph pattern.
    \item If $P$ is a graph pattern and $T$ is a finite set of variables, then (SELECT $T$ WHERE $P$) is a graph pattern.
\end{itemize}

Given the RDF graph shown in Figure \ref{fig:soa_rdf_example}, we might be interested in obtained the name of the trips of our graph. In Listing \ref{list:soa_sparql_example} we depict the SPARQL query used for obtaining the desirable result-set. The first part of the query defines the prefixes that can be used in the query. Then, the SELECT operator indicates the set of variables that will be projected from the graph pattern in query result-set (?name). Variables in SPARQL are represented using the question mark before the variable name (in the example ?name or ?trip are variables). During the query evaluation process, the variables are bound to the different values of the RDF graph that match with the triples pattern (formal definition described in~\citep{perez2009semantics}). In the example, the variable ?trip is bound to \textit{http://transport.org/trip/1 } as there are two facts in the graph that match with the two triple patterns defined in the query. Then, variable ?name is bound to ``Puerta del Sur'' and retrieved in the SPARQL result-set, usually represented in the form of a table where each column contains the values of each projected variable (see Table \ref{tab:soa_result_set}).




% Please add the following required packages to your document preamble:
% \usepackage{graphicx}
\begin{table}[h]
\centering
\caption{SPARQL result-set example}
\label{tab:soa_result_set}
\resizebox{0.2\textwidth}{!}{%
\begin{tabular}{|c|}
\hline
\textbf{?name} \\ \hline
``Puerta del Sur'' \\ \hline
\end{tabular}%
}
\end{table}

The SPARQL built-in conditions are constructed using the elements of the set ($I \cup U$) and constants. These constants can take different values such as logical connectives ($\neg,\wedge,\vee $), inequality or equality symbols ($\leq,\geq,<,>,= $) and unary predicates such as bound, isBlank, isIRI, etc.~\citep{SPARQL}. The can be combined with the aforementioned SPARQL operator such as UNION, FILTER or OPTIONAL. For example, the OPTIONAL operator has been attract many attention from the virtual knowledge graph construction field over relational databases due the difficulties to on its efficient translation to SQL~\citep{xiao2018efficient}.

In SPARQL there are many type of query forms based on graph pattern matching that allow to evaluate several query types. As we shown in Listing \ref{list:soa_sparql_example} the SELECT clause is used to select elements from the data. DESCRIBE query returns information about the resources that match a graph pattern in form of another RDF graph. CONSTRUCT is used to construct a new RDF graph based on the graph pattern indicated in the WHERE clause. Finally, ASK is a boolean-based query that retrieves true if there is at least one solution for the provided pattern.

\section{Annotations in Knowledge Graph Construction}
\label{sec:soa_annotations}
One of the main components for the construction of knowledge graphs are the annotations. Additionally to the mapping rules, that relate the target model with the input sources in a typical data integration system definition, we include in the annotations set the constraints concept. In a DIS, the constrains property allows to: i) define ad-hoc transformation functions that permit the cleaning and preparation of the input data; ii) definition of metadata annotations to describe the content of the input source. This property is essential during a knowledge graph construction process as it is able to deal with the typical features of heterogeneous data sources such as the absence of a well-defined and fixed data schema, a normalized database instance or the non-explicit declarations of relations among the sources. We start this section discussing existing approaches for the design of mappings. Then, we describe the current mapping language specifications and their standardization through the W3C. Finally, we present approaches to define, declaratively, constraints over a DIS.

\subsection{Mapping Rules}
The mapping layer contains information about how the input sources are related with the target model. There are two basic approaches for defining mapping rules in a data integration system: Local as a View (LAV) and Global as a View (GAV). In semantic web, the usual approach followed to define this rules is the Global as View one. We now provide a description of each proposals more in detail.

\subsubsection{Local as a View Mapping rules (LAV)}
In \citep{ullman1997information} the elements of the source schema $S$ are mapped  to a query $Q_G$ over the target schema $G$. The main benefits of this approach is that it supports continuous changes of the source schema (e.g., adding new sources or modify their underlying representation) since there is no need to change the query processing component. Thus, LAV is usually useful when the global schema $G$ is stable but the local schema $S$ may suffer modifications over the time. However, one of its main disadvantages is that cannot represent source $S$ information if it is not modeled in the global schema, hence, the approach usually provides partial answers for a query $Q_G$. Query translation following this approach is not a trivial process, as the $Q_G$ has to be translated to an equivalent query over the source schema $S$. These techniques are usually known as query translation using views~\citep{halevy2001answering}.

ToDo: ADD EXAMPLE

\subsubsection{Global as a View Mapping Rules (GAV)}
Global as view are defined in \citep{halevy2001answering}, where each element of the global schema $G$ is mapped to a query over $Q_S$ the source schema $S$. Opposed to LAV approach, the benefits of following a GAV approach is that it supports changes over the global schema $G$, as the queries are defined following the source schema $S$. Although there are no theoretical limitations to provide access to other data formats, ontology-based data integration processes have been traditionally focused on allowing the integration of relational databases as source schema, based on SPARQL-to-SQL translation techniques. Due to the aforementioned limitations in these techniques for LAV approaches, most of the semantic web mapping rules specifications following the GAV approach (e.g., R$_2$O, DR2Q, R2RML). 

ToDO: ADD EXAMPLE

\subsubsection{R2RML: W3C Recommendation}
Since 2012, R2RML is a W3C recommendation to declarative declare mapping rules between RDF and RDB~\citep{R2RML}. These rules are defined in an R2RML mapping document, which is formed by a set of Triples Maps (\texttt{rr:TriplesMap}). Usually, each Triple Map defines the rules for generating the entities and their properties of a defined class in the ontology and are defined as:
\begin{itemize}
    \item one logical table (\texttt{rr:LogicalTable}) that specifies the source relational table/view
    \item one subject map (\texttt{rr:SubjectMap}) that specifies how to generate the subjects of the triples and the corresponding class.
    \item from zero to multiple predicate-object maps (\texttt{rr:PredicateObjectMap}). Predicate-object map is formed by one or many predicate maps (\texttt{rr:PredicateMap}), and one to many object maps (\texttt{rr:ObjectMap}) or reference-object maps (\texttt{rr:RefObjectMap}). This last one is used when joins among logical sources are defined.
\end{itemize}

\begin{figure}[!h]
\centering
\includegraphics[width=\textwidth]{figures/state-of-the-art/R2RML-structure.pdf}
\caption{R2RML structure with its relevant properties}
\label{fig:soa_r2rml-structure}
\end{figure}

Figure \ref{fig:soa_r2rml-structure} shows the basic structure of an R2RML Triples Map, with the relations between its main properties and their cardinalities. \texttt{rr:SubjectMap}, \texttt{rr:PredicateMap} and \texttt{rr:ObjectMap} are defined as term maps (\texttt{rr:TerMap}. This property is used to generate the desirable RDF terms, either as IRIs (\texttt{rr:IRI}) Blank Nodes (\texttt{rr:BlankNode}) or literal (\texttt{rr:Literal}). The values of the term maps can be specify using the following properties: \texttt{rr:Constant} for constant values, \texttt{rr:Column} for values obtained directly from a column of a table or \texttt{rr:Template} for the ones that are a concatenation between a string and a column reference, for example, to generate subject IRIs. Furthermore, additional information can be provided such as the language of a literal, using the \texttt{rr:Language} property, or its corresponding datatype with \texttt{rr:Datatype}. Figure \ref{fig:soa_termmap-structure} gives an overview of the R2RML term map.


\begin{figure}[!h]
\centering
\includegraphics[width=0.7\textwidth]{figures/state-of-the-art/term-map.png}
\caption{R2RML \texttt{rr:TermMap} overview}
\label{fig:soa_termmap-structure}
\end{figure}

We provide a complete example of the construction of a knowledge graph through an R2RML mapping. The input table (Figure \ref{fig:soa_csv}) represents information about the stop times for a giving trip of the Madrid's metro system. It provides the corresponding identifier for the trip and each stop together with the pickup time and type. Figure \ref{fig:soa_r2rml} shows the R2RML mapping for transforming the input table to a RDF dataset following the LinkedGTFS\footnote{\url{https://lov.linkeddata.es/dataset/lov/vocabs/gtfs}} ontology. We can observe that the subject map uses a \texttt{rr:template} with reference to three different columns of the table to generate unique identifiers for the IRIs. The set of \texttt{rr:PredicateObjectMap} (POM) defines the rules to generate three different properties of the \texttt{gtfs:StopTime} class. The \texttt{gtfs:arrivalTime} POM uses the \texttt{rr:dataType} property to declare the type of the input data, \texttt{gtfs:pickupType} uses the \texttt{rr:template} to generate an uri-based literal in the output dataset, and finally, \texttt{gtfs:trip} predicate has a \texttt{rr:RefObjectMap} to generate its corresponding object, referencing to the corresponding IRI trip defined in antoher TriplesMap. Finally, the output RDF dataset is shown in Figure \ref{fig:soa_rdf}.

\begin{figure}[!h]
\centering
\includegraphics[width=0.85\textwidth]{figures/state-of-the-art/Stop_times CSV.pdf}
\caption{Excerpt of a Stop-times table}
\label{fig:soa_csv}
\end{figure}

\begin{figure}[!h]
\centering
\includegraphics[width=0.85\textwidth]{figures/state-of-the-art/R2RML-example.pdf}
\caption{R2RML mapping for stop-times table}
\label{fig:soa_r2rml}
\end{figure}


\begin{figure}[!h]
\centering
\includegraphics[width=0.85\textwidth]{figures/state-of-the-art/RDF from Stop_times.pdf}
\caption{Output RDF dataset after parsing the R2RML document}
\label{fig:soa_rdf}
\end{figure}



\subsubsection{RML: Extending R2RML for Heterogeneous Data}
The RDF Mapping Language (RML)~\citep{dimou2014rml} proposes to extend the R2RML mapping specification to cover other kinds of data formats such as CSV, XML or JSON, hence, this specification is one of the most used for constructing knowledge graph from heterogeneous data sources. In this section we explain the main differences between RML and R2RML and then we describe a new serialization of the RML mappings following a YAML syntax and its main properties~\citep{Heyvaert2018Declarative}.


\begin{table}[h]
\centering
\caption{The differences between R2RML and RML}
\label{tab:soa_rmlvsr2rml}
\begin{tabular}{l|c|c}
                                          & \textbf{R2RML}                      & \textbf{RML}                         \\
\hline                                          
\textbf{Input reference}                  & Logical Table                       & Logical Source                       \\
\hline
\textbf{Data source language}             & SQL (implicit)                      & Reference Formulation (explicit)     \\
\hline
\multirow{2}{*}{\textbf{Value reference}} & \multirow{2}{*}{column}             & Logical reference \\
                                          &                                     & (valid expression acc. \\
                                          &                                     & Reference Formulation) \\
\hline                                          
\multirow{2}{*}{\textbf{Iteration}}       & \multirow{2}{*}{per row (implicit)} & per record        \\
                                          &                                     & (explicit -- valid expression        \\
                                          &                                     & acc. Reference Formulation)
\end{tabular}
\end{table}


\noindent Summarized in Table \ref{tab:soa_rmlvsr2rml}, the main differences between R2RML and RML are:

\noindent\textbf{{Logical Source.}} A Logical Source extends R2RML's Logical Table and describes the input data source used to generate the RDF. The Logical Table is only able to describe relational databases,whereas the Logical Source defines different heterogeneous data sources, including relational databases.

\noindent\textbf{{Reference Formulation.}} As RML is designed to support heterogeneous data sources, data sources in different formats needs to be supported. One refers to data in a specific format according to the grammar of a certain formulation, which might be path and query languages or custom grammars. For example, one can refer to data in an XML file via XPath and in a relational database via SQL. To this end, the \emph{Reference Formulation} was introduced indicating the formulation used to refer to data in a certain data source.

\noindent\textbf{{Iterator.}} In R2RML it is specified that processors iterate over each row to generate RDF. However, as RML is designed to support heterogeneous data sources, the iteration pattern cannot always be implicitly assumed. For example, iterating over a specific set of objects is done by selecting them via a JSONPath expression. To this end, the \emph{Iterator} was introduced which determines the iteration pattern over the data source and specifies the extract of data used to generate RDF during each iteration. The iterator is not required to be specified if there is no need to iterate over the input data.

\noindent\textbf{{Logical Reference.}} When referring to values in a table or view of a relational database, R2RML relies on column names. However, as RML is designed to support heterogeneous data sources, rules may also refer to elements and objects, such as in the case of XML and JSON. Consequently, references to values should be valid with respect to the used reference formulation. For example, a reference to an attribute of a JSON object should be a valid JSONPath expression. To this end, (i) the \verb|rml:reference| is introduced to replace \verb|rr:column|, (ii) when a template is used, via \verb|rr:template|, the values between the curly brackets should have an expression that is valid with respect to the used reference formulation, and (iii) \verb|rr:parent| and \verb|rr:child| of a Join Condition should also have an expression that is valid with respect to the used reference formulation.


YARRRML~\citep{Heyvaert2018Declarative} is a serialization of the RML mappings based on YAML syntax~\citep{ben2001yaml}. The aim of this approaches is to help users in the creation of mapping rules using a human-readable approach. Figure \ref{fig:soa_yarrrml} shows an example of the mapping rules for constructing the KG based on the Stop Times table \ref{fig:soa_csv}. The main keys used in an YARRRML-based mapping are \texttt{prefixes} to define the used prefixes and \texttt{mappings} where the rules are specify. Each \texttt{rr:TriplesMap} is defined with a key defined by the user (stoptimes in the example), and then three main keys have to be declared: \texttt{sources} for the input sources, \texttt{s} for generating the subject and then \texttt{po} for the \texttt{rr:PredicateObjectMap} properties. In the same manner as in [R2]RML, the defined identifier for the \texttt{rr:TriplesMap} is used for the \texttt{rr:RefObjectMap} declarations (e.g., trips in Figure \ref{fig:soa_yarrrml}). Additionally, this approach provides a translator to RML\footnote{\url{https://rml.io/yarrrml/matey/}}, so any RML-compliant engine can be used when the rules are defined following this serialization.




\begin{figure}[!h]
\centering
\includegraphics[width=0.7\textwidth]{figures/state-of-the-art/YARRRML example.pdf}
\caption{YARRML mapping example based for transforming Stop Times table}
\label{fig:soa_yarrrml}
\end{figure}

\subsubsection{Mapping rules with specific features}
There are a set of mapping languages for constructing knowledge graphs that include a set of particular features. Instated of being a general solution to be applied in declarative manner to any kind of data format, there are some proposals that are focusing on providing support to concrete issues that can appear in some data formats or they extend other semantic web technologies for defining these rules such as SPARQL~\citep{SPARQL} or ShExML~\citep{prud2014shape}.  

\noindent\textbf{xR2RML.} The xR2RML proposal~\citep{michel2015translation} extends R2RML to describe mappings from common databases to RDB, focusing on NoSQL databases such as MongoDB. It also includes some of the properties from RML such as the \texttt{rml:iterator}. It contains three distinctive features to deal with issues that usually appear when hierarchical documents (XML or JSON) have to be mapped to an RDF model:
\begin{itemize}
    \item \textbf{Access to outer fields.} When a hierarchical structure has to parse to RDF, sometimes it is necessary to combine data from different levels of the tree. xR2RML incorporates the \texttt{xrr:pushDown} property with that aim. It allows the declaration of an external field outside the iterator defined, so it can be used in any part of the mapping. It is declared inside the logical source object with two properties. \texttt{xrr:reference} which defines the data reference of the value and \texttt{xrr:as} which declare the alias to be used within the mapping.  
    \item \textbf{Dynamic language tag.} xR2RML extends the \texttt{rr:language} property from R2RML and replaces it by \texttt{xrr:languageReference}. This property allows to define the language of an object dynamically, using a reference that can get the value from the database.
    \item \textbf{RDF lists and containers.} In formats such as JSON is very common to have values that are modeled as arrays of data. This kind of data structures are also allowed in RDF using \texttt{rdfs:Container} with its corresponding subclasses \texttt{rdf:Bag}, \texttt{rdf:Seq} and \texttt{rdf:Alt} when there is not order, and the \texttt{rdf:List} class is used when there is an order. xR2RML extends the term map types including \texttt{xrr:RdfList}, \texttt{xrr:RdfBag}, \texttt{xrr:RdfSeq} and \texttt{xrr:RdfAlt} that allow the construction of these data structures in RDF.
\end{itemize}

\noindent\textbf{SPARQL-Generate.} The solution proposed in~\citep{lefranccois2017sparql} presents a template-based language to construct knowledge graphs extending SPARQL 1.1. It exploits the representation capabilities from SPARQL to declare the transformation rules inside the query. The approach of using SPARQL to define the transformation rules has been also proposed in previous approaches focused on specific data formats such as Tarql for CSV files\footnote{\url{https://tarql.github.io/}}. The target users for SPARQL-Generate are knowledge graph engineers who already know SAPRQL and the target ontologies, so they do not have to learn a mapping language to perform the data integration process. However, this kind of solutions follow a procedural approach instead of a declarative one, to define the rules. Therefore, and in comparison with the rest of proposals, they cannot exploit the benefits of declarative definition of rules such as the maintainability, reproducibility and understandability.

\noindent\textbf{ShExML.} In~\citep{garcia2020shexml}, the authors proposes a solution based on the validation language for knowledge graphs, ShEx~\citep{prud2014shape}. Although it is based on this specification it uses its own syntax and grammar. In comparison to R2RML-based proposals, ShExML separates the declaration, how to extract the data from the input sources, from the shapes, how to generate the desirable RDF graph. The main objective of this proposal is to help users to create the rules, and it also provides a translation engine to transform ShExML rules to RML mappings.


\subsection{Declarative Constraints: Transformation Functions and Metadata}
Data constraints play a key role during a data integration process~\citep{cali2002data}. This property allows to validate an input dataset $D$ against an schema $S$. In previous proposals of knowledge graph construction over relational database (OBDA), they assume the existence of integrity constraints explicitly defined over the schema $S$ to propose their optimizations in SPARQL-to-SQL processes. Additionally, mapping recommendations (i.e., R2RML) for RDB2RDF approaches declare that cleaning or preparation steps are not part of the KG construction process and they have to be performed before running it. However, during the construction of a KG from heterogeneous data sources, these data may not be normalized, and information about relationships or column/key names are not always descriptive or homogeneous, among other possible issues. Hence, data consumers are usually forced to apply ad-hoc or manual data wrangling processes to consume these kind of data. In order to try to avoid manual and not reproducible cleaning/preparation steps, there are a set of declarative proposals to allow the description of constraints over data on the web. Specifically, we will be focused on two different ways to declare constraints: extensions of mapping specifications for include the possibility to define transformation functions inside the mapping rules and metadata to describe data content on the web. 

\subsubsection{Transformation Functions in Mappings}
Rahm and Do~\citep{rahm2000data} have reported the relevance of data transformations expressed with functions during data curation and integration. Grounding on this statement, different approaches have been proposed for facilitating the definition of functions to enhance data curation (e.g., \citep{galhardas2001declarative,GuptaSKGTM12,raman2001potter}). Similarly, declarative languages have been proposed to allow for the definition of functions in the mappings. An approach independent of a specific implementation context is described in~\citep{demeester2019implementation}. It enables the description, publication and exploration of functions and instantiation of associated implementations. The proposed model is the so called Function Ontology~\citep{de2016ontology} and the publication method follows the Linked Data principles. It is used as an extension over RML mapping rules to declarative allow the declaration of these transformation functions~\citep{de2017declarative}. Previous works related to this topic focus on developing ad-hoc and programmed functions. For example, R2RML-F~\citep{debruyne2016r2rml} and FunUL~\cite{junior2016funul,junior2016incorporating} allow using functions in the value of the \texttt{rr:objectMap} property, so as to modify the value of the table columns from a relational database first (R2RML-F) and other kinds of formats after (FunUL). KR2RML~\citep{slepicka2015kr2rml}, used in Karma, extends R2RML by adding transformation functions in order to deal with nested values. OpenRefine enables such transformations with the usage of GREL functions, which can be used in its RDF extension. 

ToDo: add FnO examples

\subsubsection{Metadata for data on the web}
CSV on the Web (CSVW)\footnote{https://www.w3.org/TR/tabular-data-primer/} is a W3C proposal for the definition of metadata on CSV files such as datatypes, valid values, data transformations, and primary and foreign key constraints. A related W3C proposal\footnote{https://www.w3.org/TR/csv2rdf/} defines a procedure and rules for the generation of RDF from tabular data and a few implementations that refer to this proposal are already available. 

\section{Knowledge Graph Construction Engines}
\label{sec:soa_engines}


The CSV2RDF tool is presented in~\citep{Mahmud2018}, the authors define algorithms to transform CSV data into RDF using CSVW metadata annotations, and their experimental study uses datasets from the CSVW Implementation Report\footnote{https://w3c.github.io/csvw/tests/reports/index.html}. Another tool, COW: Converter for CSV on the Web\footnote{https://csvw-converter.readthedocs.io/en/latest/} allows the conversion of datasets in CSV format and uses a JSON schema expressed in an extended version of the CSVW standard. Both are focused on RDF materialization.


\section{Evaluation of Knowledge Graph Construction}
\label{sec:soa_evaluations}

\subsection{Test-Cases and Testbeds}
In the context of Semantic Web, several specifications were recommended by W3C, such as SPARQL~\citep{SPARQL}, RDF~\citep{RDF}, SHACL~\citep{SHACL}, Direct Mapping of relational data to RDF (DM)~\citep{directMapping}, and R2RML~\citep{R2RML}. Each of these specifications has several related tools that support them. A set of test cases was defined for each one of them (SPARQL test cases\footnote{ \url{https://www.w3.org/2001/sw/DataAccess/tests/r2}}, RDF 1.1 test cases\footnote{ \url{http://www.w3.org/TR/rdf11-testcases/}}, SHACL test cases\footnote{ \url{http://w3c.github.io/data-shapes/data-shapes-test-suite/}}, and R2RML and Direct Mapping test cases\footnote{\url{https://www.w3.org/TR/2012/NOTE-rdb2rdf-test-cases-20120814/}}, respectively) that provides useful information to choose the tool that fits better to certain needs. It is also a relevant step in the standardization process of an technology or specification. We describe the R2RML in more details as it is related to the scope of this paper.

Determining the conformance of tools executing R2RML rules in the process of RDF generation is a step to provide objective information about the features of each tool. For this reason, the R2RML test cases~\citep{R2RML_test_cases} were proposed. It provides a set of 63 test cases. Each test case is identified by a set of features, such as the SQL statements to load the database, title, purpose, specification reference, review status, expected result, and corresponding R2RML rules. All the test cases are semantically described using the RDB2RDF-test\footnote{\url{http://purl.org/NET/rdb2rdf-test\#}} and Test Metadata Vocabulary\footnote{\url{https://www.w3.org/TR/2005/NOTE-test-metadata-20050914/}}. Several R2RML processors were assessed for their conformance with the R2RML specification running the test-cases. The results are available in the R2RML implementation-report~\citep{R2RML_implementation_report}. The results are also annotated semantically using the Evaluation and Report Language (EARL) 1.0 Schema\footnote{\url{https://www.w3.org/TR/EARL10/}}.

The Semantic Web community has also actively worked on the definition of several testbeds. As an example of the existing contributions, we can mention the work done in the area of federated query processing. Specifically in this area, FedBench~\citep{schmidt2011fedbench} is an exemplar benchmark; it comprises three datasets, (i.e. cross-domain, life science and SP$^2$Bench), 25 queries, and two proposed metrics to measure a federated engine performance, (i.e. total execution time and number of requests to SPARQL endpoints). LSLOD is another benchmark~\citep{hasnain2017biofed} that consists of 20 queries --classified as simple and complex; it comprises ten real-world datasets from the Life Sciences domain. LSLOD proposes to measure the performance in terms of total triple pattern-wise sources selected (TTPWSS), the number of SPARQL queries ASK, the source selection time, the overall query execution time, and the result set completeness. Finally,~\citep{montoya2012benchmarking} identify a main drawback in existing benchmarks for SPARQL federated queries; particularly, Montoya et al. focus on the study of FedBench and illustrate how the lack of considering independent variables impact on the effectiveness of the benchmark, e.g. complexity of the queries, data used, platforms involved, and endpoints. They show the relevance of these variables in order to ensure reproducibility of the results observed during an empirical evaluation. 

\subsection{Benchmarks}
Several benchmarks have been developed to measure the performance of SPARQL to SQL query translation of OBDA engine techniques. The main two proposals in this field are the Berlin SPARQL Benchmark (BSBM)~\citep{bizer2009berlin} and the Norwegian Petroleum Directorate Benchmark (NPD)~\cite{lanti2015npd}. The BSBM benchmark sets its context in the e-commerce domain, and provides a configurable data generator and a set of SPARQL queries together with their equivalent SQL queries. This benchmark has been used to compare the query performance of native RDF stores with the performance of OBDA engines that execute virtualized SPARQL access against relational databases.

%To define BSBM and NPD


Specific OBDA requirements have been analyzed by the authors of the NPD benchmark in the setting of a real-world scenario from the oil industry. The nine proposed requirements are related to the datasets, query sets, mappings and query languages. The benchmark includes a data generator, VIG~\citep{lantivig}, to generate scaled RDB instances that obtain a number of expected triples from a SPARQL query, using as inputs an ontology, an R2RML mapping document, and the schema of the RDB together with its corresponding instance. 

Simple queries defined in LSLOD~\citep{hasnain2017biofed} have been used in order to evaluate the performance of the Ontario~\citep{endris2019ontario} engine. In this work, all of the original RDF datasets were translated into RDB tables. The idea was to evaluate an heterogeneous setup consisting of RDF and RDB sources, focused on the source selection problem, and the generation of the corresponding optimized query plan. Additionally, the simple queries from LSLOD are focused on the evaluation of the distribution of star-shaped groups that do not exploit some of the features of the SPARQL language, such as FILTER, ORDER BY, GROUP BY and NOT EXISTS, which are relevant in the context of  real-life use cases.


\section{Conclusions}
\label{sec:soa_conclusions}
