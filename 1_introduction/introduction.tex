\chapter{Introduction}
\label{chap:intro}

\epigraph{Your impact will be as big as the quality of your pitch to explain the solution}{\textit{Pieter Colpaert}}

%Amount of data on the web, for example open data portals
Over the last years, a large and constant growth of data have been made available on the Web. These data are available in many different formats such as HTML tables, spreadsheets, PDF documents and web services. One of the most important and successful digital infrastructures that allows to governments, public institutions or non-profit organizations to provide a common and unique point where the data can be accessible by key stakeholders of the current societies (e.g, citizens, developers, private companies, etc.) are the open data portals. For example, at the time of writing, the European Data Portal\footnote{\url{https://www.europeandataportal.eu/catalogue-statistics/Evolution}} aggregates approximately 700K datasets from EU countries in a diversity of domains. In this context, the W3C organism has proposed a set of technologies and recommendations, which are the basis for Semantic Web, Linked Data, or the current hot topic, Knowledge Graphs. RDF~\citep{brickley1999resource} has been proposed as a standard format for data interchange on the Web, and RDF Schema~\citep{brickley2014rdf} and OWL ontologies~\citep{mcguinness2004owl} have begun to appear so as to provide shared models in some domains, while SPARQL~\citep{perez2009semantics} is defined as the standard query language for these data models. However, the amount of non-RDF data (e.g., CSV, JSON, XML) that are published in these open data portals continues to dominate the scene, and interoperability issues hinder their (re)use and consumption. 

%RDF y Semantic Web / Knowledge Graphs
Knowledge graphs, as the most relevant exponent of the use of Semantic Web technologies, have gained momentum as a result of the explosion of available data and the demand of expressive formalisms to integrate factual knowledge spread across various data sources~\citep{abs-2003-02320}. The number of hits per day of public knowledge graphs such as DBpedia\footnote{\url{https://wiki.dbpedia.org/blog/keep-using-dbpedia}} and Wikidata\footnote{\url{https://stats.wikimedia.org/}}, as well as the amount of scientific publications referencing these formal models\footnote{\url{https://pubmed.ncbi.nlm.nih.gov/?term=knowledge+graph}}, provide evidence of the spectrum of opportunities that they are bringing into the industrial and scientific landscape. Although these results endorse the success of Semantic Web technologies, also exhort the development of computational tools to scale up knowledge graphs to the astronomical data growth expected for the next years. The definition of robust methodologies able to integrate these data sources across the web is the first step that has to be solved for starting to see the web as an integrated overall database.


%Integración de datos basado en ontologias general
Data integration is not a new problem, it was already identified and addressed several decades ago with an emphasis on data in relational databases, but it is exacerbated by the availability of such data on the Web. Different techniques and tools have been used to address this problem~\citep{Lenzerini02,Halevy18}. In our work, we focus on those approaches based on ontologies and the use of Semantic Web technologies, what has been recently called knowledge graph construction (KGC). The basis of these approaches are, previously namely Ontology Based Data Access (OBDA)~\citep{poggi2008linking}, data consumers issue queries over a dataset according to a common unified view (an ontology). The relationship between the ontology and the data sources is usually available in the form of declarative mapping rules. In Ontology Based Data Integration (OBDI)~\citep{poggi2008linking}, these techniques are expanded to address heterogeneous datasets, whose data need to be integrated to provide answers to these queries. In both knowledge graph construction (KGC) approaches, two different alternatives exist to enable data access: (1) materialized KGC: where data are materialized taking into account the mappings and the ontologies (for example, data is transformed into RDF and loaded into a triple store, so that it can be queried using SPARQL), and (2) virtual KGC: where the transformation is done on the queries using the mapping rules, which can then be evaluated on the original data sources. 


%Construcción de grafos de conocimiento de fuentes heterogeneas (mappings



Several benchmarks already exist in the state of the art of virtual knowledge graph access~\cite{bizer2009berlin,lanti2015npd}, as well as in SPARQL query federation~\cite{schmidt2011fedbench,hasnain2017biofed,montoya2012benchmarking}. The VKG access BSBM benchmark~\cite{bizer2009berlin} is focused on comparing the performance of SPARQL-to-SQL query translation versus the performance of native RDF Stores, and only considers VKG engines that access relational data stores. The NPD benchmark~\cite{lanti2015npd} specifically analyzes VKG access requirements related to datasets, query sets, mapping rules and query languages. In the area of federated SPARQL engines, existing benchmarks~\cite{schmidt2011fedbench,hasnain2017biofed,montoya2012benchmarking} are tailored to the context of SPARQL endpoint federation in an homogeneous format. As a result, none of these benchmarks address the requirement of virtualized access of multiple datasets available in heterogeneous formats. Additionally, VKG engines have been evaluated in an ad-hoc manner~\cite{endris2019ontario,mami2019querying} and to the best of our knowledge, no benchmarks have been developed to evaluate these proposals in a systematic manner. 

%Contribuciones de la tesis high level







\section{Thesis Structure}
\label{sec:thesisstructure}
The reminder of the thesis is organized as follows:
\begin{itemize}
    \item In Chapter \ref{chap:soa} we analyze the current state of the art directly aligned with the topics of this work. Semantic Web and its main standard technologies for construction knowledge graphs, using declarative mapping languages and ontology-based data integration systems. We identify the limitations of the current approaches, which conduct the contributions of the presented work.
    \item In Chapter \ref{chap:objectives} we describe the objectives and main contributions of this thesis. Additionally, we present the assumptions, hypothesis and restrictions of the work.
    \item Chapter \ref{chapter:mappig-translation} defines the concept and properties of a new generation of knowledge graph construction systems, based on the idea of support semantic interoperability among the different types of declarative mapping specifications, what we call \textit{mapping translation}. Additionally, it provides a set of use cases where this idea has been already applied, with special focused on the construction of KG in the statistics domain.
    \item Chapter \ref{chapter:virtual} describes two proposals to optimize the construction of virtual knowledge graphs exploiting the concept of mapping translation. We first describe a set of optimization techniques to enhance completeness and performance of virtual knowledge graph access over tabular data. Second, we present a system that translates declarative mapping rules to programmed query endpoints for providing access to heterogeneous data sources.
    \item In Chapter \ref{chapter:benchmark} we present a comprehensive benchmark for virtual knowledge graph access, which considers multiple data formats and different data scales. Several engines from the state of the art are evaluated with this single benchmark so as to assess the current status of virtual knowledge graph access.
    \item In Chapter \ref{chapter:construction} describes a set of optimizations techniques over knowledge graph materialization approaches to provide scalability to the process. Additionally, two different studies evaluating the current status of these kinds of engines are also present.
    \item Finally, Chapter \ref{chap:conc} describes the main conclusions of this thesis and identified the future lines of research in the are of semantic data integration and knowledge graph construction.
\end{itemize}


\section{Dissemination Results}
\label{sec:disresults}

Our work has been presented in the following international workshops, conferences and research journals:

\begin{itemize}
    \item Our contribution in Chapter \ref{chapter:mappig-translation} has been published in: Oscar Corcho, Freddy Priyatna and David Chaves-Fraga: Towards a New Generation of Ontology Based Data Access, Semantic Web Journal 2020: 153-160
    \item The implementation over a real use case of the ideas proposed in Chapter  \ref{chapter:mappig-translation} has ben published in: David Chaves-Fraga, Freddy Priyatna, Idafen Santana-Pérez and Oscar Corcho: Virtual Statistics Knowledge Graph Generation from CSV files, in Proceedings of the 6th International Workshop on Semantic Statistics co-located with the 17th International Semantic Web Conference 2018 (ISWC2018). This paper received the \textbf{Best Paper Award} in the workshop.
    \item First part of our contribution to Chapter \ref{chapter:virtual} has been submitted to Semantic Web Journal: David Chaves-Fraga, Edna Ruckhaus, Freddy Priyatna, Maria-Esther Vidal and Oscar Corcho: Enhancing Virtual Ontology Based Access over Tabular Data with Morph-CSV. We are waiting for the corresponding reviews.
    \item Second part of our contribution to Chapter \ref{chapter:virtual} has been published in: David Chaves-Fraga, Freddy Priyatna, Ahmad Alobaid and Oscar Corcho: Exploiting Declarative Mapping Rules for Generating GraphQL Servers with Morph-GraphQL. International Journal of Software Engineering and Knowledge Engineering, 2020: 785-803 and in: Freddy Priyatna, David Chaves-Fraga, Ahmad Alobaid, and Oscar Corcho: morph-GraphQL: GraphQL Servers Generation from R2RML Mappings, Proceedings of the 31st International Conference on Software Engineering \& Knowledge Engineering (SEKE2019).
    \item Chapter \ref{chapter:benchmark} has been published in: David Chaves-Fraga, Freddy Priyatna, Andrea Cimmino, Jhon Toledo, Edna Ruckhaus and Oscar Corcho: GTFS-Madrid-Bench: A benchmark for virtual knowledge graph access in the transport domain. Journal of Web Semantics 2020.
    \item The first contribution to Chapter \ref{chapter:construction} has been published in: Enrique Iglesias, Samaneh Jozashoori, David Chaves-Fraga, Diego Collarana, and Maria-Esther Vidal: SDMRDFizer: An RML interpreter for the efficient creation of rdf knowledge graphs, Proceedings of the 29th ACM International Conferrence on Information and Knowledge Management 2020 (CIKM2020).
    \item Second contribution to Chapter \ref{chapter:construction} has been published in: Samaneh Jozashoori, David Chaves-Fraga, Enrique Iglesias, Maria-Esther Vidal and OScar Corcho: FunMap: Efficient Execution of Functional Mappings for Knowledge Graph Creation, Proceedings of the 19th International Semantic Web Conference 2020 (ISWC2020).
    \item Third and fourth contributions to \ref{chapter:construction} have been  published correspondingly in: Pieter Heyvaert, David Chaves-Fraga, Freddy Priyatna, Oscar Corcho, Erick Mannens, Ruben Verborgh and Anastasia Dimou: Conformance test cases for the RDF mapping language (RML), Proceedings of 1st Iberoamerican Knowledge Graphs and Semantic Web Conference 2019 (KGSW2019); and in: David Chaves-Fraga, Kemele M. Endris, Enrique Iglesias, Oscar Corcho and Maria-Esther Vidal: What are the Parameters that Affect the Construction of a Knowledge Graph?. In OTM Confederated International Conferences On the Move to Meaningful Internet Systems 2019 (OTM2019).
\end{itemize}



