\chapter{Introduction}
\label{chap:intro}

\epigraph{Your impact will be as big as the quality of your pitch to explain the solution}{\textit{Pieter Colpaert}}

\section{Thesis Structure}
\label{sec:thesisstructure}
The reminder of the thesis is organized as follows:
\begin{itemize}
    \item In Chapter \ref{chap:soa} we analyze the current state of the art directly aligned with the topics of this work. Semantic Web and its main standard technologies for construction knowledge graphs, using declarative mapping languages and ontology-based data integration systems. We identify the limitations of the current approaches, which conduct the contributions of the presented work.
    \item In Chapter \ref{chap:objectives} we describe the objectives and main contributions of this thesis. Additionally, we present the assumptions, hypothesis and restrictions of the work.
    \item Chapter \ref{chapter:mappig-translation} defines the concept and properties of a new generation of knowledge graph construction systems, based on the idea of support semantic interoperability among the different types of declarative mapping specifications, what we call \textit{mapping translation}. Additionally, it provides a set of use cases where this idea has been already applied, with special focused on the construction of KG in the statistics domain.
    \item Chapter \ref{chapter:virtual} describes two proposals to optimize the construction of virtual knowledge graphs exploiting the concept of mapping translation. We first describe a set of optimization techniques to enhance completeness and performance of virtual knowledge graph access over tabular data. Second, we present a system that translates declarative mapping rules to programmed query endpoints for providing access to heterogeneous data sources.
    \item In Chapter \ref{chapter:benchmark} we present a comprehensive benchmark for virtual knowledge graph access, which considers multiple data formats and different data scales. Several engines from the state of the art are evaluated with this single benchmark so as to assess the current status of virtual knowledge graph access.
    \item In Chapter \ref{chapter:construction} describes a set of optimizations techniques over knowledge graph materialization approaches to provide scalability to the process. Additionally, two different studies evaluating the current status of these kinds of engines are also present.
    \item Finally, Chapter \ref{chap:conc} describes the main conclusions of this thesis and identified the future lines of research in the are of semantic data integration and knowledge graph construction.
\end{itemize}


\section{Dissemination Results}
\label{sec:disresults}

Our work has been presented in the following international workshops, conferences and research journals:

\begin{itemize}
    \item Our contribution in Chapter \ref{chapter:mappig-translation} has been published in: Oscar Corcho, Freddy Priyatna and David Chaves-Fraga: Towards a New Generation of Ontology Based Data Access, Semantic Web Journal 2020: 153-160
    \item The  David Chaves-Fraga, Freddy Priyatna, Idafen Santana-Pérez and Oscar Corcho: Virtual Statistics Knowledge Graph Generation from CSV files, in Proceedings of the 6th International Workshop on Semantic Statistics co-located with the 17th International Semantic Web Conference 2018 (ISWC2018). This paper received the \textbf{Best Paper Award} in the workshop.
    \item First part of our contribution to Chapter \ref{chapter:virtual} has been submitted to Semantic Web Journal: David Chaves-Fraga, Edna Ruckhaus, Freddy Priyatna, Maria-Esther Vidal and Oscar Corcho: Enhancing Virtual Ontology Based Access over Tabular Data with Morph-CSV. We are waiting for the corresponding reviews.
    \item Second part of our contribution to Chapter \ref{chapter:virtual} has been published in: David Chaves-Fraga, Freddy Priyatna, Ahmad Alobaid and Oscar Corcho: Exploiting Declarative Mapping Rules for Generating GraphQL Servers with Morph-GraphQL. International Journal of Software Engineering and Knowledge Engineering, 2020: 785-803 and in: Freddy Priyatna, David Chaves-Fraga, Ahmad Alobaid, and Oscar Corcho: morph-GraphQL: GraphQL Servers Generation from R2RML Mappings, Proceedings of the 31st International Conference on Software Engineering \& Knowledge Engineering (SEKE2019).
    \item Chapter \ref{chapter:benchmark} has been published in: David Chaves-Fraga, Freddy Priyatna, Andrea Cimmino, Jhon Toledo, Edna Ruckhaus and Oscar Corcho: GTFS-Madrid-Bench: A benchmark for virtual knowledge graph access in the transport domain. Journal of Web Semantics 2020.
    \item The first contribution to Chapter \ref{chapter:construction} has been published in: Enrique Iglesias, Samaneh Jozashoori, David Chaves-Fraga, Diego Collarana, and Maria-Esther Vidal: SDMRDFizer: An RML interpreter for the efficient creation of rdf knowledge graphs, Proceedings of the 29th ACM International Conferrence on Information and Knowledge Management 2020 (CIKM2020).
    \item Second contribution to Chapter \ref{chapter:construction} has been published in: Samaneh Jozashoori, David Chaves-Fraga, Enrique Iglesias, Maria-Esther Vidal and OScar Corcho: FunMap: Efficient Execution of Functional Mappings for Knowledge Graph Creation, Proceedings of the 19th InternationalSemantic Web Conference 2020 (ISWC2020).
    \item Third and fourth contributions to \ref{chapter:construction} have been  published correspondingly in: Pieter Heyvaert, David Chaves-Fraga, Freddy Priyatna, Oscar Corcho, Erick Mannens, Ruben Verborgh and Anastasia Dimou: Conformance test cases for the RDF mapping language (RML), Proceedings of 1st Iberoamerican Knowledge Graphs and Semantic Web Conference 2019 (KGSW2019); and in: David Chaves-Fraga, Kemele M. Endris, Enrique Iglesias, Oscar Corcho and Maria-Esther Vidal: What are the Parameters that Affect the Construction of a Knowledge Graph?. In OTM Confederated International Conferences On the Move to Meaningful Internet Systems 2019 (OTM2019).
\end{itemize}



