
\begin{acknowledgementslong} 

Realizar un trabajo de investigación complejo, como es el caso de una tesis doctoral, no se puede entender si no como un trabajo colaborativo. Es por ello, que a lo largo de las lineas que siguen trataré agradecer y mencionar a todo aquel que ha contribuído, de una forma u otra, en el desarrollo de esta investigación.

Para comenzar, esta tesis habría sido imposible de realizar sin el apoyo y soporte continuo de dos personas que admiro profundamente desde el primer momento en el que las conocí y a las que considero, mis dos padrinos acádemicos, Oscar Corcho y Maria-Esther Vidal. Oscar, gracias por haberme dado la posibilidad de realizar este trabajo bajo tu supervisión en el OEG. Además de haber disfrutado enormemenete con las discusiones que hemos ido teniendo a lo largo de estos cinco años, me gustaría agradecerte la confianza depositada en mi y la libertad que me has otorgado, quizá un poco difícil de gestionar y comprender al comienzo, pero que sin ella, no habría crecido personal y profesionalmente de la manera que lo he hecho. Al final, el tiempo pone a cada uno en el lugar que le corresponde. Gracias por tu paciencia y continua positividad, que me han ayudado a \textit{``ver siempre el lado bueno de la vida''} (como cantaban los \textit{Monthy Python} en ``La Vida de Brian''), tan necesario en los momentos complicados y los tiempos que corren. La pasión, el creer en lo que uno hace y el valor del trabajo de investigación lo aprendí de Maria-Esther. Nunca sabré cómo agradecerte el trato recibido, como uno más de tus estudiantes, el tiempo que estuve de estancia en SDM y por todo el trabajo que hemos realizado juntos durante la segunda parte de mi doctorado. Es complicado expresar en tan pocas lineas, el completo y desinteresado apoyo que he recibido por tu parte, así como los valores y enseñanzas sobre el esfuerzo, la perfección en el trabajo y tu entusiasmo en todo lo que haces. Simplemente, gracias. 

El sentimiento de pertenencia a un grupo es otra de las claves para poder llegar a completar una tesis con éxito. Más que un grupo, creo que el Ontology Engineering Group es una familia, un gran familia. Las enriquecedoras discusiones y charlas que he tenido con mis ``hermanos'', Paola y Carlos, sobre el doctorado, la investigación y la vida en general. Gracias al resto de doctorandos: Alba, Serge, Elvi, Pablo, Julia, etc, a los ``seniors'' del OEG y ex-OEGs Esteban, Raul Alcazar, Raul García, JARG, Ana Ibarrola, Elena, Miguel Ángel, Maria Póveda, Idafen, Jose Luis, etc. y a Álvaro, mi colega de festivales y cervezas. Pero en especial agradecer el apoyo a Patricia, mi compi de batallas en la vida, de viajes en el coche, de charlas interminables, del ``siempre hay que decir si a todo'', de las idas y venidas y las noches madrileñas hasta el amanecer. Asimismo, ha sido un gusto compartir tiempo y experiencias con todos los integrantes del grupo de integración de datos. Especialmente me gustaría agradecer a Edna su ayuda en los artículos que hemos escrito juntos y su paciencia en la revisión esta tesis. Freddy, gracias por tus ideas locas y tu buen humor. Jhon, Ana, Luis, Dani, Julián, Ahmad, Marlene y Andrea he disfrutado enormemente discutiendo, definiendo y logrando retos con todos vosotros. 

To the people from Ghent, specially to Pieter and Anastasia, my ``postdocs'' in the shadows. Thanks for giving me the possibility to work and collaborate with you and your teams. Hope that this is only a promising beginning. Lucie and Sam, I am extremely glad to meet you, first in Bertinoro, and then in Hannover. Lucie you were my support during the good and bad moments of my internship in Germany. I am really happy to have a friend like you in my life. Sam, I have never going to forget our legendary week in Rodhes and all the awesome things that come after it, I really admire your passion and I enjoyed to work with you. Finally, to the rest of the SDM-TIB family: Enrique, Kemele, Ahmad, Ariam, Maria Isabel...

E por último, pero non menos importante, quedariame agradecer a miña xente de sempre, os meus galegos. Meus pais que forón unha continua axuda durante todo esta longa viaxe, deles quédome coa sua forza para afrontar os momentos delicados e a paixón coa que enfrentan o día a día. Mamá, gracias por todas as discusións e o teu punta de vista sobre como debe ser un bo investigador, eres unha referente para min. Papá, ogallá algún día poida desfrutar de cada momento cao mesma intesidade e felicidade coa que o fas tí. Martín, o meu irmán, creo que ainda que en diferentes etapas da vida, durante estes anos, os nosos camiños se xuntaron pouco a pouco, e o teu apoio incondicional sempre foi de gran axuda. Gracias os meus amigos de sempre: Rafa, Marta, Clau, Paula, Erle, Ánxela, Olga e Minia... Pero tamén ós que me atopei en Madrid: Santi, Patricia, Pamela, Carolina...

Las última lineas van dedicadas a a




\end{acknowledgementslong}




