
\begin{acknowledgementslong} 

Realizar un trabajo de investigación complejo, como es el caso de una tesis doctoral, no se puede entender sino como un trabajo colaborativo. Por ello,  a lo largo de las líneas que siguen trataré de agradecer y mencionar a todo aquel que ha contribuido, de una forma u otra, al desarrollo de esta investigación.

Para comenzar, la realización de esta tesis habría sido imposible sin el apoyo constante de dos personas que admiro profundamente desde el primer momento en que las conocí y a las que considero mis dos padrinos académicos, Oscar Corcho y Maria-Esther Vidal.  Gracias, Oscar, por haberme dado la posibilidad de realizar este trabajo bajo tu supervisión en el Ontology Engineering Group (OEG). Además de disfrutar enormemente con las discusiones que hemos ido teniendo a lo largo de estos cinco años, me gustaría agradecerte la confianza que depositaste en mí y la libertad que me has otorgado, quizá un poco difícil de gestionar y comprender al comienzo por mi parte, pero sin la que  no habría crecido personal y profesionalmente de la manera que lo he hecho. Al final, el tiempo pone a cada uno en el lugar que le corresponde. Gracias por tu paciencia y continuo optimismo, que me han ayudado a \textit{``ver siempre el lado bueno de la vida''} (como cantaban los \textit{Monthy Python} en ``La Vida de Brian''), tan necesario en los momentos complicados y, particularmente, en los tiempos que corren. De Maria-Esther admiro su pasión por el trabajo; con ella aprendí a creer en lo que uno hace,  a valorar el trabajo de investigación. Nunca sabré cómo agradecerte el trato recibido, como si fuese uno más de tus estudiantes, durante mi estancia en SDM, así como tu contribución en todo el trabajo que hemos realizado juntos durante la segunda parte de mi doctorado. Es complicado expresar en tan pocas líneas el completo y desinteresado apoyo que he recibido por tu parte, así como los valores y enseñanzas sobre el esfuerzo, la perfección en el trabajo y tu entusiasmo por todo lo que haces. Simplemente, gracias. 

El sentimiento de pertenencia a un grupo es otra de las claves para poder llegar a completar una tesis con éxito. Más que un grupo, creo que el OEG es una familia, una gran familia. Esta tesis no habría sido lo mismo sin las enriquecedoras discusiones y charlas que he tenido con mis ``hermanos'', Paola y Carlos, sobre el doctorado, la investigación y la vida en general. Gracias al resto de doctorandos (o doctores ya): Alba, Serge, Elvi, Pablo, Julia, etc, a los ``seniors'' del OEG y ex-OEGs: Esteban, Raúl Alcazar, Vicky, Raúl García, JARG, Ana Ibarrola, Elena, Miguel Ángel, Paco, Maria Poveda, Idafen, José Luis, etc. Gracias a Juan y Alvaro, dos personas con las que he compartido mucho (también cervezas) durante estos años y que me han ayudado a madurar; me llevo conmigo dos grandes amistades. Especialmente, quiero agradecer su apoyo a Patricia, mi ``compi' de batallas en la vida, de viajes en el coche, de charlas interminables, del ``siempre hay que decir sí a todo'', de las idas y venidas, y las noches madrileñas hasta el amanecer. Asimismo, ha sido un gusto compartir tiempo y experiencias con todos los integrantes del grupo de integración de datos. Me gustaría agradecer también a Edna su ayuda en los artículos que hemos escrito juntos y su paciencia durante la revisión de esta tesis. Freddy, gracias por tus ideas locas y tu buen humor. Jhon, Ana, Luis, Dani, Julián, Ahmad, Marlene y Andrea, he disfrutado enormemente discutiendo, definiendo y logrando retos con todos vosotros. 

To the people from Ghent, especially to Pieter and Anastasia, my ``postdocs'' in the shadows. Thanks for giving me the possibility to work and collaborate with you and your teams, especially to Ben, Pieter and Julián. Hope that this is only a promising beginning. Lucie and Sam, I am extremely glad to meet you, first in Bertinoro, and then in Hannover. Lucie, you were my support during the good and bad moments of my internship in Germany. I am really happy to have a friend like you in my life. Sam, I am never going to forget our legendary week in Rodhes and all the awesome things that came after it. I really admire your passion and I enjoyed working with you; you will be a reference researcher for many people. Finally, to the rest of the SDM-TIB family: Enrique, Kemele, Ahmad, Ariam, Maria Isabel...

E por último, pero non menos importante, quero dar as grazas  a miña xente de sempre, aos meus galegos. A meus pais, que foron unha continua axuda durante  esta longa viaxe; deles quédome coa súa forza para afrontar os momentos delicados e coa paixón coa que confrontan o día a día. Mamá, grazas por todas as discusións e o teu punto de vista sobre que é un bo investigador; eres un referente para min. Papá, ogallá algún día poda desfrutar de cada momento coa mesma intensidade e ledicia coa que o fas ti. Martín, meu irmán, creo que aínda que en diferentes etapas da vida, durante estes anos os nosos camiños xuntáronse pouco a pouco, e o teu apoio incondicional sempre foi de gran axuda. Gracias aos meus amigos de sempre: Rafa, Marta, Clau, Paula, Erle, Ánxela, Carme, Polly, Olga e Minia... Pero tamén aos que atopei en Madrid: Santi, Patricia, Pamela, Carolina... Sempre conseguiron sacarme un sonriso e fixéronme ver que a academia e a universidad non son o único que existe. Nada sería igual de non ser por eles.

Las últimas líneas van dedicadas a toda esa gente que me ha ayudado, muchas veces sin saberlo, a conseguir esta meta. Las largas mañanas de experimentación con 180º y Virginia Díaz, las tardes-noches de tenis con Rafa, Iñaki y Andrés, entre otros; la última temporada escribiendo la tesis en el piso de Madrid y saliendo a ``rodar'' unos kilómetros con Ismael; gracias a todos los estudiantes que han participado en alguna edición del Open Summer of Code Spain. Gracias, Jorge, por tu hospitalidad y amistad durante mi estancia en Santiago de Chile.

Por encima de todo, gracias por hacerme sentir en casa.

\vspace{10mm}

\textit{
\null\hfill Vuelvo a casa \\
\null\hfill De la zozobra de mi corazón \\
\null\hfill Ahora vivo aquí \\
\null\hfill Pensé que estaba solo y descubrí \\
\null\hfill Que estaban todos los que importan \vspace{6mm} \\
} 
\null\hfill Iván Ferreiro - Casa, ahora vivo aquí

\end{acknowledgementslong}









